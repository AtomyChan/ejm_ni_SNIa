\documentclass{article}

\begin{document}
\title{$H_0$ from NIR observations of SNIa}
\maketitle
The following is a brief summary of the steps in the analysis to estimate $H_0$

\begin{itemize}

\item Using the peak magnitude and $H_0$ from Folatelli+2010, I calculate the zero point for the given filter using equation \ref{eq:zp} (this value is independent of $H_0$ since both terms on the RHS depend on $H_0$)

\item The $H_0$ value is calculated using equation \ref{eq:h0}

\item For $M_{max}$ I use the model values provided (without the faintest model since we know that 91bg-likes are fainter even in the NIR). 

\item Using the mean and error on the zero point and the mean and standard error on the mean on $M_{max}$ from the models, I create 10000 realisation of $H_0$ (similar to Cartier+2014)

\item The final estimate is the mean and standard deviation of the resulting $H_0$ values

\item For the systematic uncertainty, I propagate a 0.03 mag uncertainty in the zero point measurement. 

\end{itemize}


\begin{equation}
\label{eq:zp}
ZP= 25 - log(H_0)/0.2 + M_{peak} 
\end{equation}



\begin{equation}
\label{eq:h0}
H_0 = 10^{0.2*(M_{max}-ZP+25)}
\end{equation}


\end{document}
