In our sample, we observe a strong correlation between the $M_{Ni}$ and $t_2$ in $Y$ and $J$, and less so in the $H$ band. 
This provides us with direct evidence that the timing of the second maximum is governed by the amount of Nickel produced by the supernova
since it leads to a later ionization transition of the iron group elements at late time (mainly, $^{56}Co$) from doubly to singly ionized \citep{Kasen2006}.

This trend is confirmed by a strong correlation between  $t_L$  and $M_{Ni}$ indicating that objects with more Ni produced have a slower
rate of reddening and the Fe and Co lines appear later in the spectrum, which delays the onset of the lira law phase, and also the second maximum.

This relation offers great insight into measuring the $M_{Ni}$ for objects not in the low-reddening sample, but with extensive NIR data. 
A striking example of this application is the nearby SN2014J in M82, which is heavily occluded by host galaxy dust. Since this prevents 
an accurate measurement of $M_{Ni}$ from the bolometric light curves and there is a large disparity in the different values published in the literature
using this method, we use the relations we obtain to constrain the $M_{Ni}$. For SN2014J, we have a unique opportunity to compare different estimation methods, 
since its proximity has allowed $\gamma$ ray Co line detection and therefore, another extinction independent measurement of the $M_{Ni}$. Our value of 0.58 $\pm$ 0.21 $M_{odot}$
compares very well with \citet{Churazov2014}, who find $M_{Ni}$ of 0.61 $\pm$ 0.13 $M_{odot}$. The brightness of SN2014J at late times, due to its proximity, permits us to obtain
NIR spectra at $\sim$ 300 days, which can provide an accurate measurement of the extinction and therefore, an accurate $M_{Ni}$ from the bolometric light curve. This presents
us with a confrontation of several different methods to measure the $M_{Ni}$ and hence obtain a conclusive estimate on the amount of Ni produce in this SN.

Since $\gamma$ detections are unlikely for farther out SN and most of them are too faint at $\sim$ +300 days for IR spectroscopy, we apply our method to other heavily reddened SN
that are farther away than SN2014J. The first object we analyse is SN2006X. From the measurement of 0.57 $\pm$ 0.15 $M_{odot}$, we conclude that
2006X produced the average amount of Ni for an SNIa.

We also analyse the bolometric light curves at peak and during the late phase of exponential decline. We find that the SN in our sample have a uniform 
late time bolometric decline rate, indicating that the internal structure of the ejecta is similar for most SN. This confirms the deductions from the optical and NIR 
light curves in previous studies and from the bolometric light curves in sample of C00. We also find that the bolometric light curves, unlike the optical light curves,
have a narrow distribution of the $\Delta m_{15}$ parameter.  

We conclude from our findings that there is a strong dependence of the $t_2$ and the colour evolution (parametrized by $t_L$) on the $M_{Ni}$