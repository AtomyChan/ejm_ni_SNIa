The flux emitted by an SNIa in the UV, optical and NIR traces the radiation converted from the radioactive decays of newly synthesized isotopes.
As the SN emits most of its flux in the UV to NIR passbands, the "uvoir bolometric flux" represents a physically meaningful quantity \citep{Suntzeff1996}

We select a low-reddening sample so that our measurements are less sensitive to a reddening law. 
For objects with sufficient amount of near maximum data in the optical and the NIR, we construct UBVRIJH bolometric light curves. We do not use $K$ band data since there are very few objects in the sample with well-sampled $K$ band light curves. For objects with well-sampled $K$ light curves we calculate the flux emitted in the $K$ band and find that it is between $1-3 \%$, thus, not using the $K$-band is not a dominant source of uncertainty. 
The magnitudes were corrected 
for reddening using a CCM reddening law for each filter. The values for the extinction are presented in table \ref{tab:mni}. The uncertainty in the reddening estimate
was propagated into the calculation of the bolometric flux
Using zero-points in the given filters, the magnitudes were converted to fluxes. 
The resulting light curve, in ergs/$cm^2$/s  was converted into an absolute bolometric light curve 
by using the distances of the SN derived from the host galaxy redshift. 

Since all distances are scaled to an $H_0=70 km s^{-1} Mpc ^{-1}$the errors in the luminosity distance are only affected by the relative errors in the 
distance moduli (see Table \ref{tab:mni} for values and uncertainty estimates). For objects not in the hubble flow, we use distance measurements from published estimates (which use others methods eg. Cepheid, Tully-Fisher relation etc.). 

The bolometric light curves were interpolated using a cubic spline. 
In order to get an $L_{Bol}(max)$ we required sampling in the individual bands at pre-maximum epochs. Thus, for objects without NIR coverage before $B_{max}$, we use the UBVRI light curves. 
The errors on the peak were calculated from the errors in the fluxes of the bolometric maximum using a Monte Carlo for 1000 realisations of the light curve.  
%\subsection{Deriving $M_{Ni}$ from $L_{max}$}
%Since our final aim is to derive the $M_{Ni}$

\begin{table*}
\caption{The sample of SNe which have low reddening, as defined in the text. The references for the data are presented along with the extinction values and the distanes used to calculate the bolometric light curves  {\bf to add: distance modulus values }}

\begin{center}
\begin{tabular}{llcccrrr}
\hline
SN  & $\mu$ & $e_{\mu}$  & $E(B-V)_{host}$ & $E(B-V)_{MW}$ & Filters \\
\hline
SN2008gp	&	0.66	&	0.14		&	$0.098(0.022)$	&	0.104(0.005)	& UBVRIJH	\\
%SN2008bq	&	0.78	&	0.23	&	1.18	&	$0.136(0.027)$	&	0.077(0.002)	&	\\
SN2007as	&	0.44	&	0.05		&	$0.050(0.011)$	&	0.123(0.001)	& UBVRI	\\
SN2008bc	&	0.68	&	0.19		&	$<0.019$	&	0.225(0.004)	& UBVRIJH	\\
%SN2004gs	&	0.37	&	0.05	&	0.54	&	$0.148(0.024)$	&	0.026(0.001)	&	\\
SN2008hv	&	0.49	&	0.13		&	$0.074(0.023)$	&	0.028(0.001)	& UBVRIJH	\\
SN2008ia	&	0.6	&	0.14		&	$0.066(0.016)$	&	0.195(0.005)	& UBVRIJH	\\
SN2005na	&	0.79	&	0.24		&	$0.061(0.022)$	&	0.068(0.003)	& UBVRI	\\
SN2005eq	&	0.73	&	0.2		&	$0.044(0.024)$	&	0.063(0.003)	& UBVRIJH	\\
%SN2006D		&	0.57	&	0.08	&	0.93	&	$0.134(0.025)$	&	0.039(0.001)	&	\\
SN2005M		&	0.76	&	0.08		&	$0.060(0.021)$	&	0.027(0.002)	& UBVRIJH	\\
SN2007on	&	0.3	&	0.09		&	$<0.007$	&	0.010(0.001)	& UBVRIJH	\\
SN2007nq	&	0.58	&	0.17		&	$0.046(0.013)$	&	0.031(0.001)	& BVRI	\\
SN2005am	&	0.48	&	0.2		&	$0.053(0.017)$	&	0.043(0.002)	& UBVRIJH	\\
SN2005hc	&	0.8	&	0.2		&	$0.049(0.019)$	&	0.028(0.001)	& UBVRIJH	\\
%SN2005ke	&	0.13	&	0.1	&	0.26	&	$0.263(0.033)$	&	0.020(0.002)	&	\\
SN2004gu	&	0.74	&	0.15		&	$0.096(0.034)$	&	0.022(0.001)	& BVRI	\\
SN2011fe	&	0.52	&	0.15		&	$0.03 (0.01)$	&	0.021(0.001)	& UBVRIJH	\\
SN2001ba	&	0.58	&	0.15		&	$ 0.06 (0.02)$  &     0.08 (0.002)	& UBVRIJH	\\
SN2002dj	&	0.64	&	0.26		&	$   0.04 (0.03)$ & 0.06 (0.003)		& UBVRIJH		\\
SN2002fk	&	0.74	&	0.23		&	$0.07 (0.02)$   & 0.02 (0.003)		& UBVRIJH 				\\
SN2008R		&	0.25	&	0.1		&  $0.009(0.013)$ & 0.062(0.001)          & 	UBVRIJH		\\
SN2005iq	&	0.52	&	0.11		& $0.040(0.015)$ & 0.019(0.001) & UBVRIJH	\\	
SN2005ki	&	0.51	&	0.27		& $0.016(0.013)$ & 0.027(0.001) & UBVRIJH	\\
SN2006bh	&	0.42	&	0.15		& $0.037(0.013)$ & 0.023(0.001) & UBVRIJH	\\
SN2007bd	&	0.6	&	0.13		&  $0.058(0.022)$ & 0.029(0.001)        & UBVRIJH	\\
\hline
\end{tabular}
\label{tab:lr}
\end{center}
\end{table*}

\iffalse
\subsubsection{Interpolating DDC models}

From these bolometric light curves, we derive $M_{Ni}$ values by interpolating the relation between log $L_{bol}(max)$ and $M_{Ni}$ from the DDC models of \citet{Blondin2013}
For objects without NIR coverage near maximum, we interpolate the values for the synthetic pseudo-bolometric light curves 
calculated only using the UBVRI filters. 


\subsubsection{Arnett's rule}




To determine the $M_{Ni}$, we use a simple relation between the amount of Ni produced and the luminosity at maximum, given by \eqref{eq:arn}.


The constant factor relating the $M_{Ni}$ to the peak luminosity is calculated using a rise time of 19 days and the error corresponds to an error of 3 days.

\fi

For objects with no NIR coverage near maximum, we apply a correction like in \citet{Stritzinger2006} and increase the $M_{Ni}$ value by 1.1. In \citet{Blinnikov2006}, the authors found that using a UVOIR light curve with the correction for the NIR, Arnett's rule estimates the $M_{Ni}$ to $\leq$ 0.05 $M_{\odot}$.
