\documentclass{article}
\begin{document}
Bivariate linear regression fits usin sklearn/statsmodels return a very high standard error on the parameter estimates $a_j$, $a_y$ and c
where
\begin{equation}
L_{max}=a_j*t2_j+a_y*t2_y+c
\end{equation}

moreover, the y-band slope is -0.0072. A \emph{negative} slope implies a lower $L_{max}$ for a higher $t_2(Y)$ which is clearly not what the data suggests. The high standard error also means that the combined fit yields very high errors on $L_{max}$. This points to the problem of multicollinearity. We look for possible evidence for this and how to solve the problem
{\bf Evidence for multicollinearity}

\begin{itemize}
\item	Insigficant t-stat for the parameters, despite a hgih F-stat
\item	high $r^2$ between $x_1$ and $x_2$
\item	high standard error and opposite direction of the parameters
\item	VIF (variance inflation factor) $>>$ 10
\end{itemize}
where VIF is used as a 'rule of thumb' statistic to test the collinearity of predictor variables
It is related to the pearsonr coefficient as 
\begin{equation}
VIF=\frac{1}{1-r^2}
\end{equation}

In order to verify the parameter estimates, we looked at the output parameters from different least squares packages, which all yielded a negative slope for the $t_2(Y)$


{\bf Possible solutions}

The above criteria for testing the presence of multicollinearity as seen in our dataset. We,therefore look at the possible solutions for this condition, without dropping a variable (that is another possibility which is recommended in such cases, but since the aim is to see how much better than a $J$-band only fit we can do, we keep this as a last resort).

1. Partial Least Squares: Using the linear model package in scikit learn, we then look at the parameter estimates from a partial least squares. The problem of directionality still remains 

2. Ridge Regression: This method uses an l2 regularization with a linear least squares. The parameter estimates are the same.

We find that none of the given solutions are adequate to give the desired parameter estimates. A PCA is an alternative, but it would reduce the dataset to a lower dimensionality, an equivalent of dropping one of the correlated regressors.






\end{document}
